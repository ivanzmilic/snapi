%\documentclass{aa}
\documentclass[referee]{aa}

\usepackage{color}

%\bibpunct{(}{)}{;}{a}{}{,};
\usepackage{amsmath}
\usepackage{natbib}
\usepackage{graphicx}
%------------------------------------------------------------------------
%Added by language editor to facilitate margin notes.
%------------------------------------------------------------------------
%\usepackage{marginnote}
\usepackage{color}
\setlength{\marginparwidth}{40mm}
\setlength{\marginparsep}{5mm}
\newcommand{\aamarginnote}[1]{{\boldmath$\color{red}\bigvee$}
\marginpar{\baselineskip3ex{\color{red}#1}}}
%------------------------------------------------------------------------
\begin{document}

   \title{Line response functions in non-local thermodynamic equilibrium}

   \subtitle{Unpolarized case}

   \author{I. Mili\'{c}\inst{1,2}
   \and
   M. van Noort\inst{1}}
   
   \institute{Max-Planck Institut f\"{u}r Sonnersystemforschung, Justus-von-Liebig-Weg 3, 37077 G\"{o}ttingen, Germany\\
   \email{milic@mps.mpg.de; vannoort@mps.mpg.de}
         \and
         Astronomical observatory Belgrade, Volgina 7, 11060 Belgrade, Serbia\\
             }

   \date{}
   \titlerunning{Response functions in NLTE}
   \authorrunning{I. Mili\'{c} \& M. van Noort}

% \abstract{}{}{}{}{} % 5 {} token are mandatory

  \abstract
  % context heading (optional)
   {Response functions provide us with quantitative measure of sensitivity of emergent spectrum to perturbations in solar atmosphere and are thus the method of choice for interpretation of spectropolarimetric observations. For the lines formed in the solar chromosphere, it is necessary to compute these responses taking into account non-local thermodynamic equilibrium (NLTE) effects.}  
  % aims heading (mandatory)
  {We demonstrate how to analytically compute the response of the level populations to the perturbation of a given physical quantity at given depth in the atmosphere, in the NLTE approximation. These perturbations are then used to compute opacity and emissivity perturbations which are further propagated to obtain the response of emergent intensity.}
   % methods heading (mandatory)
   {Our method is based on the derivative of the rate equations, where we explicitly incorporate spatial coupling in the radiative rate terms. After taking account of all the co-dependencies, the problem reduces to a linear system of a dimension equal to the product of number of spatial points and number of energy levels.}
  % results heading (mandatory)
   {We compare analytically computed response functions with ones obtained using finite difference approach and find very good agreement. As an intermediate result, we developed a new, more precise way of propagating opacity and emissivity perturbations through numerical solution of radiative transfer equation.} 
  % conclusions heading (optional), leave it empty if necessary
   {This method enables fast evaluation of the response of the emergent spectrum to perturbations of given quantity at given depth and enables more efficient NLTE inversions.}
 
   \keywords{Line: formation; Radiative transfer;}

   \maketitle

%___________________________________________________________________________________

\section{Introduction}

Response functions of spectral lines \citep{Mein71, Beckers75, Landi77} describe the sensitivity of an emergent Stokes spectrum to the perturbations of different physical parameters at different points in the atmosphere. They are useful in the forward modeling approach, to propose and analyze various diagnostics \citep{Han06}, but their main strength and application is in derivative-based spectral line fitting \citep[in solar physics widely known as ``inversion'', see for example][]{SIR}. Given the perturbation of an atmospheric parameter (temperature, velocity, magnetic field), the computation of the response function in practice reduces to finding the perturbations of opacity and emissivity and then propagating these perturbations using the formal solution of radiative transfer equation. It is intuitively clear that finding the perturbations of opacity and emissivity actually reduces to finding the perturbations of atomic and molecular number densities as well as perturbations of the populations of individual levels. 
%We will try to make this dependence clear and explicit in the remainder of the paper.

In the approximation of local thermodynamical equilibrium (LTE), computing perturbations to level populations is relatively straightforward \citep[see, for example][]{SIR, dtibook}, as all the relevant number densities depend only on the local quantities. In addition, opacity and emissivity are related through Kirchoff law. LTE is a good approximation for spectral lines formed deep in the photosphere, where transitions between the levels are dominated by collisions. However, for the lines which are formed in upper photosphere and the chromosphere (e.g. H$\alpha$, Ca\,II infrared triplet, Ca\,II H\&K, Mg h\&k lines...), this approximation is far from valid as the transition rates are dominated by radiation field. This introduces both spatial and nonlinear coupling between level populations. NLTE problem and its solutions are extensively discussed in the literature and probably the best reference is a classical monograph by \citet{Mihalasbook} \citep[or a recent re-issue by][]{SAbook2014}.

Response functions in NLTE approximation, however, have not been studied in great detail. The first attempt, by \citet{HectorI}, was based on so called FDC (fixed departure coefficients) approximation which essentially computes NLTE response functions by scaling LTE ones with the departure coefficients (departure coefficient is ratio of a level population in NLTE to the LTE one). In that paper authors find that this is rather crude approximation and resort to computing response functions numerically (i.e. using finite differences). In the papers of \citet{Hector_halpha} and \citet{Han06}, response functions are computed numerically. Also in the NLTE inversion code NICOLE \citep{NICOLE}, responses, and thus the $\chi^2$ derivatives to model parameters, are computed in a numerical way.

In this paper we demonstrate how the response functions of level populations to atmospheric perturbations can be computed in an analytical way, with low computational cost. From these, it is straightforward to compute response functions of the emergent intensity. As an intermediate step, we show how to, given perturbations of opacity and emissivity, compute very precisely perturbations of intensity throughout the atmosphere. In Section 2 we revisit the concept of response functions. In Section 3 we outline the method for analytical computation of NLTE functions and show the explicit derivation for a case of pure line transfer. In Section 4 we compare analytically computed response functions with ones computed using finite difference method and discuss the discrepancies. Finally we conclude by discussing possible applications of our approach and future steps. 

\section{Concept of response functions}
\label{concept}

We assume that the emergent radiation is formed in a one-dimensional atmosphere represented by a discrete set of points at which the values of atmospheric parameters are given. The relevant parameters for spectral line formation are: temperature, pressure, macroscopic velocity, microturbulent velocity and the magnetic field. Given the values of these parameters, we can obtain opacity and emissivity of the medium at each spatial and wavelength-point and in each direction. These then enter the radiative transfer equation (RTE), which in 1D axisymmetric, time-independent, polarization-free case reads:
\begin{equation}
 \mu \frac{dI(z,\mu,\lambda)}{dz} = -\chi(z,\mu,\lambda) I(z,\mu,\lambda) + \eta(z,\mu,\lambda)
 \label{rte}
\end{equation}
where $z$ is geometrical coordinate along the atmospheric normal, $\mu = \cos \theta$, where $\theta$ is the angle with respect to the atmospheric normal, and $\chi$ and $\eta$ are unpolarized (i.e. scalar) opacity and emissivity. Given the boundary conditions, (usually numerical) solution of the radiative transfer equation then yields the emergent spectrum which is to be compared with, or fitted to the observed one. Note that emergent intensity depends on the opacity and emissivity throughout the whole atmosphere (i.e. radiative transfer immediately implies non-locality). 

We define the response function of the emergent intensity to the perturbation of a given atmospheric quantity $q$ at depth point $k$ in the following way:
\begin{equation}
 R_{q_k} \equiv \frac{d I_0(\mu,\lambda)}{d q_k}.
\end{equation}
This response function (derivative), can then be expressed using derivatives of the emergent intensity to opacity and emissivity:
\begin{align}
 \frac{d I_0(\mu,\lambda)}{d q_k} = & \sum_l^{ND} \frac{\partial I_0(\mu,\lambda)}{\partial \chi_l(\mu,\lambda)} \frac{d \chi_l(\mu,\lambda)}{d q_k} \nonumber \\
 + & \sum_l^{ND} \frac{\partial I_0(\mu,\lambda)}{\partial \eta_l(\mu,\lambda)} \frac{d \eta_l(\mu,\lambda)}{d q_k}.
\end{align}
Where $ND$ is number of depth points in the atmosphere. This formulation takes into account that perturbation $q_k$, in the general case, changes opacity and emissivity in the whole atmosphere, not only at point $k$. Now, $\frac{\partial I_0(\mu,\lambda)}{\partial \chi_l(\mu,\lambda)}$ and $\frac{\partial I_0(\mu,\lambda)}{\partial \eta_l(\mu,\lambda)}$ depend on the exact numerical scheme used to solve RTE. We discuss this computation in more detail in appendix\,\ref{Appendix_A} and, for the moment, assume that we are able to compute $\frac{\partial I_0}{\partial \chi_l}$ and $\frac{\partial I_0}{\partial \eta_l}$ (we are omitting angle and wavelength dependence, which should be clear from the context).

The next step is to compute the derivatives of opacity and emissivity everywhere with respect to $q_k$. For the moment, we restrict the discussion to line processes and relax this assumption later. In the case of pure line transfer:
\begin{align}
 \chi_l(\mu,\lambda) = \frac{h\nu}{4\pi}(n_{l,j} B_{ji} - n_{l,i}B_{ij}) \phi^{ij}_l(\mu,\lambda) 
\end{align}
where $i$ and $j$ are indices of upper and lower level of the transition, $n_{i,j}$ denotes number density of atoms in state $i$ at spatial point $l$, $B_{ji}$ and $B_{ij}$ are Einstein coefficients of absorption and stimulated emission, respectively and $\phi^{ij}_l(\lambda)$ is the line absorption profile. Similarly:
\begin{align}
 \eta_l(\mu,\lambda) = \frac{h\nu}{4\pi}n_{l,i}A_{ij} \phi^{ij}_l(\mu,\lambda) 
\end{align}
where $A_{ij}$ is Einstein coefficient of spontaneous emission. Here we have assumed that line emission and absorption profiles are identical (so called complete frequency redistributon - CRD), which is a good approximation for most of the spectral lines of interest. Exception are very strong lines formed very high in the atmosphere, but even there CRD describes well the line core.

The derivative of the opacity with respect to $q_k$ is then:
\begin{align}
  \frac{d \chi_l(\mu,\lambda)}{d q_k} = & \frac{h\nu}{4\pi}(n_{l,j} B_{ji} - n_{l,i}B_{ij}) \frac{d\phi^{ij}_l(\mu,\lambda)}{d q_k} \nonumber \\
 + & \frac{h\nu}{4\pi}(\frac{d n_{l,j} B_{ji}}{d q_k} - \frac{d n_{l,i}B_{ij}}{d q_k}) \phi^{ij}_l(\mu,\lambda).
 \label{chi_derivative}
\end{align}
Similar expression can be written for the derivative of the emissivity. We thus see that the derivative (response) of the opacity/emissivity can be expressed through the derivatives of the level populations and the derivative of local absorption profile. The latter is determined by the temperature, microturbulent velocity and the number density of collisional partners (neutral hydrogen, electrons...). For the following, we assume that electron density is either constant or that its response can be approximated with LTE value. In principle the problem can be generalized and we can find NLTE responses of electron density as well, but we leave that discussion for following papers. We will therefore assume that we can find the derivative of the absorption profile and that it is strictly local, that is:
\begin{equation}
 \frac{d\phi^{ij}_l(\mu,\lambda)}{d q_k} = \delta_{lk} \frac{d\phi^{ij}_k(\mu,\lambda)}{d q_k}.
\end{equation}
What remains to be computed are the responses (derivatives) of the level populations. In approximation of non-local thermodynamic equilibrium this is actually the most complicated part of the computation. Once the level populations are known, one can compute the perturbations of opacity and emissivity in the atmosphere and then, depending on the particular form of the numerical formal solution, propagate these perturbations to obtain the response of the emergent intensity.

%\subsection{Local thermodynamic equilibrium approximation}
%In the approximation of LTE number densities of atoms and molecules, ionization states and level populations are given by Saha-Boltzmann statistics. Individual level populations can be computed as 
%\begin{equation}
%n_i = \frac{N}{Z(T)} g_i e^{-E_i/kT} 
%\label{Boltzmann}
%\end{equation}
%where $N$ is the total population of given ionization state, $Z$ is partition function of given atom in given ionization state, $g_i$ is the statistical weight of level $i$ and $E_i$ is energy of level with respect to the ground state. Ion number density is obtained using Saha and chemical equilibrium equations. In practice, this means solving a linear system where unknowns are logarithms of number densities of individual ions and molecules. In practice, for a modest amount of species, the solution of this, so called, chemical equilibrium, takes very little compared to line synthesis. Therefore, the derivative of $N$ can be found easily in numerical way, using finite differences. From there the individual perturbations of level populations follow. The computations of LTE response functions then looks like this:
%\begin{enumerate}
% \item Solve the chemical equilibrium and from there obtain the derivatives of populations of individual ions.
% \item Using Eq.\,\ref{Boltzmann}, compute the responses of level populations.
% \item Substitute these back in Eq.\,\ref{chi_derivative} and compute the derivative of the opacity, and, in the similar way, of the emissivity.
% \item Depending on the formal solver for the numerical solution of RTE, propagate these perturbations to obtain response function of emergent intensity (see Appendix\,\ref{Appendix_A}).
%\end{enumerate}
%Process like this is in the core of modern LTE inversion codes like SIR\,\citep{SIR} or SPINOR\,\citep{SPINOR}. We want to make two important points on the derivatives of level populations: i) level populations, in this approximation, do not depend on microturbulent nor on systematic velocity. It is only directly influenced by temperature, and indirectly by the magnetic field as it can influence pressure stratification of the atmosphere; ii) all the number densities depend strictly on local quantities. That is, the perturbation of temperature at point $k$ cannot perturb level populations in point other then $k$.

%Intuitively, we can say that this will change in NLTE case, as the level populations also depend on the radiation field. This will make both of the two above remarks invalid. Radiation field depends on the velocities in the atmosphere, and, so to the populations. Also, radiation field introduces non-locality, which means that now perturbations of physical parameters in one point are able to change level populations in other points as well. We will formalize this in the following section, and then show how to compute NLTE response functions.

\subsection{Level population responses in NLTE approximation}
\label{nlte_lvls}

%Now we allow for the radiative transition between the levels. For simplicity, we assume that bound-free and free-bound processes are still governed by collisions (this assumption can be relaxed, and we will demonstrate how). We also assume that only source of opacity and emissivity are non-overlapping spectral lines. 


%A method of choice for handling this coupling is so called ALI (accelerated/approximate lambda iteration). For this work we opt for the approach by \citet{RH1}, except that for the formal solution we use second-order Bezier splines as explained in \citet{JaimeBezier}. For an excellent review of ALI-related methods, see, for example \citet{Hubeny03}.

In the approximation of non-local thermodynamical equilibrium (NLTE), the level populations are determined both by the local temperature and by the (non-local) radiation field. The level populations follow from the so called statistical equilibrium equation which, for the population of level $i$ at depth point $l$, reads:
\begin{equation}
 \frac{d n_{l,i}}{d t} = \sum_j (n_{l,j} T_{l,ji} - n_{l,i} T_{l,ij}) = 0.
 \label{SE}
\end{equation}
Here $T_{ij}$ is total rate of transitions from level $i$ to level $j$. Generally, $T_{ij} = C_{ij} + R_{ij}$ where $C$ stands for collisional and $R$ for radiative transitions. Deeper in the photosphere, collisional rate is much higher than the radiative one and thus LTE is recovered. The situation is opposite in the chromosphere where radiative collisions dominate. Furthermore, lines we are interested in are optically thick, so it is necessary to treat full problem in NLTE and self-consistently solve radiative transfer and statistical equilibrium equations (Eqs.\,\ref{rte} and \ref{SE}). At least from the theoretical point of view, this problem can be considered ``solved'' mostly thanks to the application of so called ``Accelerated Lambda Iteration'' methods \citep[for an insightful review see][]{Hubeny03}.  We want to go one step further and compute not only the populations but also their responses (derivatives). For the brevity, we use following notation:
\begin{equation}
 {\mathcal R}_{lik} \equiv \frac{d n_{l,i}}{d q_k}
\end{equation}
and refer to it as ``response functions of level populations.'' To compute them, we start by differentiating Eq.\,\ref{SE} with respect to perturbation of $q_k$:
\begin{equation}
 \sum_j \left ({\mathcal R}_{ljk} T_{l,ji} - {\mathcal R}_{lik} T_{l,ij} + n_{lj} \frac{d T_{l,ji}}{d q_k} - n_{li} \frac{d T_{l,ji}}{d q_k} \right ) = 0.
 \label{rateder1}
\end{equation}
System in Eq.\,\ref{SE} is undetermined, so the last equation is usually replaced with:
\begin{equation}
 \sum_j n_{lj} = N_l,
\end{equation}
which, after taking the derivative becomes:
\begin{equation}
 \sum_j {\mathcal R}_{ljk} = \frac{d N_l}{d q_k}.
\end{equation}
Where $N_l$ is the total number density of species in question. For the simplicity's sake we will assume that this derivative is strictly local and that it can be computed directly from chemical equilibrium equations. To solve Eq.\,\ref{rateder1} for $\mathcal{R}_{lik}$, we need the derivative of the rates:
\begin{equation}
\frac{d T_{l,ji}}{d q_k} = \frac{d C_{l,ji}}{d q_k} + \frac{d R_{l,ji}}{d q_k}.
\label{rateder2}
\end{equation}
Let us discuss the equations given above. If we neglect the radiative rates both in Eq.\,\ref{SE} and Eq.\,\ref{rateder1} and solve for populations and population responses, we end up with LTE values. In this specific case, it is actually easier to find level populations, as well as their responses, directly from Saha-Boltzmann equations \citep[see, e.g.][]{SIR}. 

Next step is to account for radiative rates in Eq.\,\ref{SE}, but keep only collisional rates (and their derivatives) in Eq.\,\ref{rateder1}. This would correspond to FDC (fixed departure coefficients) of \citet{HectorI}. They have used following approximation for population responses:
\begin{equation}
{\mathcal R}_{lik} = b_{l,i}  {\mathcal R}^*_{lik}
\end{equation}
where quantities with asterisk in superscript denote LTE values and $b_{l,i}$ is the so-called departure coefficient (ratio of NLTE to LTE populations). It is easy to see why \citet{HectorI} find poor agreement between FDC approximation and numerically computed responses in the upper layers. Firstly, the dependence of the departure coefficient on $q_k$ is not taken into account. Maybe more important is the fact that spatial coupling is completely neglected and the responses are assumed to be strictly local, which they are not. The effect of non-locality is mostly pronounced in the upper, more dilute layers of the atmosphere. Also non-local effects are more important the finer the spatial grid is. Obviously, to get fully consistent response functions, we need to account for radiative rates and their derivative. While radiative rates themselves are known, as we have already solved NLTE problem, finding the derivative of the radiative rates is more complicated.

\subsection{Derivative of radiative rates}

For simplicity, we assume that the continuum (i.e. bound-free and free-bound) processes are governed by collisional rates only (this assumption can and will be relaxed later). Radiative transition rates from bound level $i$ to bound level $j$, have the following form:
\begin{equation}
 R_{l,ij} = A_{ij} + B_{ij} J_{l,ij}
\end{equation}
Where $A_{ij} \equiv 0$ when $i<j$ and $J_{l,ij}$ is the so called scattering integral for the transition $i\rightarrow j$ at depth point $l$:
\begin{equation}
 J_{l,ij} = \frac{1}{2} \int_{-\infty}^{\infty} \phi_{l,ij}(\lambda) r_{\lambda}\,d\lambda \int_{-1}^{1} I_l(\mu,\lambda)\,d\mu. 
 \label{JJ}
\end{equation}
Where $r_{\lambda}$ is ratio of line opacity to total opacity, which we for the moment take to be identical to one. Taking the derivative of Eq.\,\ref{JJ} we get:
\begin{align}
 \frac{d J_{l,ij}}{d q_k} = \frac{1}{2} \int_{-\infty}^{\infty} \int_{-1}^{1} ( & \frac{d I_l(\mu,\lambda)}{d q_k} \phi_{l,ij}(\lambda) + %\nonumber \\
 I_l(\mu,\lambda) \frac{d \phi_{l,ij}}{d q_k} )\,d\lambda\,d\mu
 \label{JJ2}
\end{align}
The second term in the integrand, involves only taking the derivative of the line profile, is strictly local and can be computed with relative ease. We define its integral as:
\begin{equation}
\Phi_{l,ij} = \frac{1}{2} \int_{-\infty}^{\infty} \int_{-1}^{1} I_l(\mu,\lambda) \frac{d \phi_{l,ij}}{d q_k}\,d\lambda\,d\mu.
\end{equation}
By expanding the derivative of the local intensity through the derivative of opacity and emissivity, we get:
\begin{align}
\frac{d I_l(\mu,\lambda)}{d q_k} = & \sum_{l'} \sum_{i'} ( \frac{\partial I_l(\mu,\lambda)}{\partial \chi_l'(\mu,\lambda)} \frac{\partial \chi_l'(\mu,\lambda)}{\partial n_{l',i'}} \nonumber \\
+& \frac{\partial I_l(\mu,\lambda)}{\partial \eta_l'(\mu,\lambda)} \frac{\partial \eta_l'(\mu,\lambda)}{\partial n_{l',i'}} ) {\mathcal R}_{l'i'k}
+  \sum_{l'} \sum_{i'} \sum_{i''<i'}  p_{l',i'i''}
\end{align}
where 
\begin{equation}
 p_{l',i'i''} = \left ( \frac{\partial I_l(\mu,\lambda)}{\partial \chi_l'(\mu,\lambda)} \frac{\partial \chi_l'(\mu,\lambda)}{\partial \phi_{l',i'i''}} + 
\frac{\partial I_l(\mu,\lambda)}{\partial \eta_l'(\mu,\lambda)} \frac{\partial \eta_l'(\mu,\lambda)}{\partial \phi_{l',i'i''}} \right ) \frac{d \phi_{l',i'i''}}{d q_k}
\end{equation}
describes the influence of changes of the line profile in the transition $i'i''$ at depth point $l'$ on the intensity at depth point $l$. Concisely written, response of specific intensity to perturbation in $q_k$ can be written as:
\begin{equation} 
 \frac{\partial I_l(\mu,\lambda)}{\partial q_k} = \sum_{l'} \sum_{i'} a_{ll'ii'} {\mathcal R}_{l'i'k} + \sum_{l'}\sum_{i'}\sum_{i''} p_{l',i'i''}.
 \label{derI_concise}
\end{equation}
Now, substituting Eq.\,\ref{derI_concise} back into Eq.\,\ref{JJ2}, and integrating over angles and wavelengths, we get:
\begin{equation}
 \frac{\partial J_{l,ij}}{\partial q_k} = \sum_{l'} (A_{ll'ij} {\mathcal R}_{ljk}+ A_{ll'ii} {\mathcal R}_{lik}) + \sum_{l'} P_{l',ij} + \Phi_{l,ij}
\end{equation}
where $A$ and $P$ correspond to angle and wavelength integrated quantities $a$ and $p$, respectively. Finally, substituting this back into the response of the statistical equilibrium equation (Eq.\,\ref{rateder1}), we get, for number density of state $i$ at depth point $l$:
\begin{align}
&\sum_j \left [ T_{l,ji}\mathcal{R}_{ljk} - T_{l,ij}\mathcal{R}_{lik} + (n_{l,j}B_{ji} - n_{l,i}B_{ij}) \sum_{l'} (A_{ll'ij} {\mathcal R}_{ljk}+ A_{ll'ii} {\mathcal R}_{lik}) \right ]  = \nonumber \\
& = \sum_j (n_{l,i} \frac{d C_{l,ij}}{d q_k} - n_{l,j} \frac{d C_{l,ji}}{d q_k}) + \sum_j \Phi_{l,ij} + \sum_j \sum_{l'} P_{l',ij}. 
 \label{final_linear_system}
\end{align}
Left hand side of equation \ref{final_linear_system} contains all the co-dependencies between the responses of levels. It is crucial to notice that coefficients in front of $\mathcal R_{lik}$ do not depend neither on the depth of the perturbation ($k$), nor on the type of the quantity we are perturbing (temperature, velocity, magnetic field...). Right hand side contains all other perturbations which explicitly affect the rates. In the case of pure line transfer, the three contributors are:
\begin{itemize}
 \item Derivative of the collisional rates which is, in first approximation strictly local (i.e. responds only to perturbation in point $l$). Also, it is reasonable to assume that collisional rates respond only to temperature and pressure.
 \item Derivative of the local line profile which influences the integration over wavelengths and angles to obtain the scattering integral. This factor is local as well and depends practically on all the atmospheric parameters (temperature, pressure, velocity, magnetic field).
 \item Derivative of line profile at other points $l'$, which influences opacity and emissivity in points $l'$, which then in turn influence intensity $I_l$ through the transfer of radiation. This factor also depends on all the atmospheric parameters.
\end{itemize}
In the general case, bound-free transitions are included in the statistical equilibrium equation, and there are other contributors to opacity and emissivity (e.g. electron and Rayleigh scattering, bound-free, free-free and H- opacity, etc.). Right hand side of the Eq.\,\ref{final_linear_system} then contains all the perturbations to the rate equations which can be computed explicitly, while the left hand side contains the coupling coefficients between all levels at all depths. So the system has the form
\begin{equation}
 \hat{a} \vec{x} = \vec{b},
\end{equation}
where $\vec{x}$ are level population responses, $\hat{a}$ describes level co-dependencies and $\vec{b}$ known responses of other quantities. This is a linear system of equations with $ND \times NL$ unknowns, where $ND$ is number of depth points and $NL$ is number of levels in the considered atom. As matrix $\hat{a}$ does not depend on $q_k$, it can be computed once, decomposed (for example, using LU decomposition), and then used to solve linear system and obtain level responses to various perturbations at various depth points. 

\subsection{Advantages of analytical response functions}

Alternative to computing the response functions analytically is to do it numerically. That is, compute:
\begin{equation}
 R_{q_k} \equiv \frac{d I_0(\mu,\lambda)}{d q_k} = \frac{I_0(\mu,\lambda; \vec{q} - \Delta q_k/2) - I_0(\mu,\lambda; \vec{q}+\Delta q_k/2)}{\Delta q_k}
 \label{fin_diff}
\end{equation}
for a very small $\Delta q_k$. Each computation of $I_0$ requires one solution of NLTE problem. There are two major problems related to this approach:
\begin{enumerate}
 \item If we want to compute response to perturbations at each of the points, we need $\mathcal{O}(ND)$ NLTE solutions, which takes much longer than solving the linear system from Eq.\,\ref{final_linear_system}. If we are somehow parameterizing our atmosphere using reduced number of parameters (as, for example, when using nodes in depth-dependent inversions), this problem is mitigated a bit but it still requires an order of, say, 10 NLTE solutions, compared to just one + solution of the linear system for the responses.
 \item NLTE problem is highly non-linear, and thus to obtain precise response functions $\Delta q_k$ should be very small. However, that also means that NLTE solution needs to be as close to the ``true'' one as possible (leaving aside the question of what is true emergent spectrum from a model atmosphere, as different numerical schemes yield different results), which means that NLTE solutions must be converged to much higher precision than usually. This is not necessary for the analytical computation as it will work with any solution taken to be ``true''.
\end{enumerate}

It is important to stress that our analytical method is not limited to perturbations in only one point. As we said already, left hand side of Eq.\,\ref{final_linear_system} does not depend on perturbation in question. We can thus easily perturb different quantities at different depths and from there construct, so to speak, equivalent r.h.s. and then solve linear system to obtain corresponding responses. This makes our approach suitable for implementation in inversion codes, as a perturbation of a parameter in a single node perturbs the value of given parameter in several atmospheric points.

That being said, we reflect on the drawbacks. The most serious one is that fully-consistent approach in case where radiative transitions are not governed only by line radiation can become relatively complicated. Consider the case in which the opacity is not influenced only by line opacity of the element in question, but also by, say, bound-free opacity of some other element. If we take the latter to be in LTE, then this coupling is not problematic as all known sources of perturbation can be grouped on the r.h.s of Eq.\,\ref{final_linear_system} and responses can be found with relative ease. If there are two elements in NLTE (which can be the case, if we are treating, for example, hydrogen and an additional element), they will necessary become coupled and we have to solve for the responses of all NLTE levels in all elements simultaneously. There will be cases where neglecting this coupling will be good approximation, as there will also be the cases where it will be a quite bad one. However, we are to expect that for very strong lines this coupling can be neglected. Formally, the same coupling is also valid for computing the derivative of electron density, which then influences shapes of line profiles, b-f and f-f transitions, and collisional line transitions, which basically means whole rate matrix.   

\section{Results}

For the purpose of testing we have implemented above procedure in the computer code, which solves NLTE radiative transfer problem using ALI formalism of \citet{RH1}. We have chosen a short characteristics based formal solution which uses second order Bezier splines for the interpolation of the source function, as described by \citet{JaimeBezier}. While this choice of formal solution brings much-needed accuracy and stability to approximate operator-based solution of the NLTE problem, it causes some complication in the explicit computation of $\partial I_l/\partial \chi_{l'}$ and $\partial I_l/\partial \eta_{l'}$, on which we will comment in detail in Appendix\,\ref{Appendix_A}.

Our main aim is to test the level responses computed with the analytical approach versus ones computed numerically (that is, using Eq.\,\ref{fin_diff}). We start with a simple example and then work our way up toward more complicated ones. 

%For readers' sake and more insight in each of the problems, we compute the emergent intensity responses as well. Note that in Appendix\,\ref{Appendix_A} we remark in more detail on the computation of intensity responses which we use to compute emergent intensity responses.

%\subsection{Two level atom in homogeneous slab}

%We start with the simplest possible example: a two-level atom with one continuum level. Continuum is considered to be populated by collisions only. This means that continuum is populated according to LTE and follows LTE responses, which also allows us to compare precision of using this approach to compute LTE and NLTE responses. Line is formed in a finite, one dimensional, homogeneous, isothermal slab of temperature equal to 5000\,K. Geometrical discretization is double-logarithmic with finer spacing toward the edges of the slab. The geometrical thickness of the slab and number density of atoms is tuned so the optical depth in the center of line is approximately $10^4$. Line profile is taken to be constant (i.e. not influenced by the temperature). This basically means that perturbations in the atmosphere only change the collisional rates, which cause ``direct'' perturbations in the statistical equilibrium equation, which further propagate through the atmosphere. Ionization energy corresponds to hydrogen's (13.6\,eV), energy of the lower level is taken to be zero, while upper level has excitation energy of $2.479$\,eV($\approx 5000\,\rm{\AA}$). Photon destruction probability is taken to be $\varepsilon \approx 10^{-4}$ ($\varepsilon = C_{\rm ul} / (A_{\rm ul} + C_{\rm ul})$). The slab is spatially discretized using 31 points, which is rather coarse for this optical depth. This, however, allows us to test both stability of our NLTE solver, as well as robustness of our approach to computing analytical response functions.

%\begin{figure*}
% \includegraphics[width = 0.5\textwidth]{figs/2_lvl_slab_spectrum.eps}
% \includegraphics[width = 0.5\textwidth]{figs/2_lvl_slab_responses.eps}
% \includegraphics[width = 0.5\textwidth]{figs/2_lvl_slab_2d_differences.eps}
% \includegraphics[width = 0.5\textwidth]{figs/2_lvl_slab_analytical_responses_intensity.eps}
% \caption{Top left: Emergent line profile; Top right: Relative responses to local perturbations. Points are numerically computed values while lines are analytical ones; Bottom left: Relative differences (in percent) between level responses computed in numerical and analytical way; Bottom right: Relative response function of emergent intensity given in percent. See the text for the discussion.}
% \label{example_1}
% \end{figure*}
 
%Fig.\,\ref{example_1} shows emergent line profile and responses we are interested in. The line is seen in emission with deep self-reversal. This is typical for lines formed in finite slab with no incident radiation. Note that this is a quite typical NLTE example, where the medium is both scattering dominated everywhere ($A_{ul} \gg C_{ul}$), optically thick at some wavelengths, and optically thin in other ones. We then show the responses of level populations to local perturbations in temperature. Unless stated otherwise we plot the relative responses, which means change of population (or intensity, or other relevant quantity) for the unit perturbation (in this case 1\,K), divided by unperturbed quantity. We chose to plot the local responses firstly because it is easier to see agreement (or lack thereof) between numerical and analytical responses, but also, even in the case of NLTE, local responses (that is, the response of the population in point $l$ to perturbations in point $l$) are higher and more important than non-local ones (we remark further on this in the following subsection, where we consider more realistic atmosphere). Judged by eye, the agreement between numerically and analytically computed responses is excellent. We see that the relative response of the lower level is practically zero, as either way the most of the electrons are in the ground state. The response of the continuum (i.e. ionized state) is constant throughout the slab because we have assumed that it is governed by collisional processes only. As the slab is completely homogeneous, the responses of the continuum are constant. Finally, we see that response of the upper level of the line is both non-negligible ($\approx 10^-3\,\rm K^{-1}$), and that it locally decreases toward the edge of the slab. Reason for that is, of course, increase of the effects of the non-locality as we move toward the edges where the geometrical (and thus optical depth) spacing is finer. 

%We then show the relative difference between analytical and numerical responses for the upper level of the transition for each combination of points (i.e. $\mathcal R_{lik}$ for $i=2$, and each combination of $l$ and $k$). We see that the relative difference is under $1\%$ everywhere and in most cases it is actually much smaller than that. Finally, for reader's sake, we show the relative responses of the emergent intensity to perturbations in the atmosphere. We note that for this specific case, analytically and numerically computed intensity response function are indistinguishable. Also, unless stated otherwise, we will plot intensity response function normalized with respect to emergent intensity at given wavelength, that is:
%\begin{equation}
% \bar{R}(h_k,\lambda) = \frac{R(h_k,\lambda)\lambda)}{I_0(\lambda)}
%\end{equation}



\subsection{Pure line transfer - prototype line}

\subsubsection{Two-level atom}

In the first example we consider a very simple model for the formation of H$\alpha$ line in semi-empirical FALC model of solar atmosphere\citep{FALC}. The line is modeled as a two level atom, with levels corresponding to second and third level of hydrogen. We assume that population of the ionized state is fixed to the LTE value, which means that NLTE effects are only redistributing electrons between the two levels of the line. Emissivity and opacity are due to the line absorption and emission only. We focus, for the moment, on the responses of level populations and emergent intensity to temperature. Our main aim is to compare responses computed in analytical way with ones obtained using FDC approximation and ones computed using finite differences. We will always plot ``relative'' responses, that is, the responses of a certain quantity divided by the quantity itself. To show the response of the emergent intensity, we plot:
\begin{equation}
 \bar{R}(h_k,\lambda) = \frac{R(h_k,\lambda)\lambda)}{I_0(\lambda)}
\end{equation}

\begin{figure*}
 \includegraphics[width = 0.5\textwidth]{figs/h_alpha_2lvl_numerical_responses_intensity.eps}
 \includegraphics[width = 0.5\textwidth]{figs/h_alpha_2lvl_analytical_responses_intensity.eps}
 \includegraphics[width = 0.5\textwidth]{figs/h_alpha_2lvl_fdc_responses_intensity.eps}
 \includegraphics[width = 0.5\textwidth]{figs/h_alpha_2lvl_relative_difference_responses_intensity.eps}
 \caption{Intensity response functions for a two-level atom line in FALC model atmosphere. Top left: Numerical (finite difference) computation of intensity responses; Top right: Responses computed analytically using the method explained in the paper; Bottom left: Responses computed using FDC approach; Bottom right: Relative differences, normalized with respect to the maximum response,  between analytical and numerical computations (in log scale).}
 \label{example_1}
 \end{figure*}
 
Response functions to temperature, computed using three different approaches are shown in Fig.\,\ref{example_1}. For the numerical computation we have used a perturbation of $10\,\rm{K}$ and converged the NLTE problem down to the relative change of $10^{-6}$. 

The line intensity response functions resemble those computed by \citet{Hector_halpha} (note that the authors there are plotting absolute, normalized response functions, so negative response cannot be seen). We see that FDC (bottom left) severely overestimates the response to temperature at heights around 1500\,km. Reason for this is that these are computed using same assumptions as LTE response functions. That is, the population responds strictly locally and the response is governed by Saha-Boltzmann statistics. In extreme examples of NLTE lines (such is H$\alpha$), this is a very bad approximation as the level populations are essentially set by the radiation coming from lower layers. It can be seen that response function computed analytically using method we have described (top right), is visually indistinguishable from the ones computed using finite differences. At bottom-right panel of Fig.\,\ref{example_1} we show the differences between the response functions computed in analytical and numerical way, divided by maximum absolute value of the numerical response function, that is:
\begin{equation}
r(h,\lambda) = \frac{\bar{R}^{\rm num}(h_k,\lambda) - \bar{R}^{\rm an}(h_k,\lambda)}{\rm max (|\bar{R}^{\rm num}(h_k,\lambda)|)}.
\end{equation}
$r(h,\lambda)$ is smaller then $0.1\%$ everywhere in the atmosphere. This is a level of agreement which does not only qualitatively describe line formation process but also enables use of the analytically response function in NLTE inversions. 

Speed-wise, the implementation of the analytically computed response functions saves a lot of compuational time. For illustration, the time needed to set up, compute and de-compose the matrix on the left hand side of Eq.\,\ref{final_linear_system} is, for our specific implementation, smaller then the one full NLTE solution. Note again that for the model described with $N_P$ parameters, one needs an order of $N_P$ full NLTE solutions to numerically compute the response to all the parameters.


\subsubsection{Multilevel atom}

We, very briefly report the results for the 4-level hydrogen-like atom. The atom consists of a ground level which is assumed to be in LTE and 3 NLTE levels with atomic constants corresponding to leves 2-4 of hydrogen. Again, we only consider line processes. The difference with respect to the previous problem is that non-linearity is now more pronounced as we are dealing with a multilevel atom. This now allows us to test the stability of NLTE solver and to analyze the behavior of our approach. Fig.\,\ref{example_2} shows response function of H$\beta$ line computed with this approach. Agreement is, again, very good, and smaller than few parts in thousand everywhere in $(\lambda,h)$ plane.  

\begin{figure*}
 \includegraphics[width = 0.5\textwidth]{figs/h_beta_3lvl_numerical_responses_intensity.eps}
 %\includegraphics[width = 0.32\textwidth]{figs/h_beta_3lvl_analytical_responses_intensity.eps}
 \includegraphics[width = 0.5\textwidth]{figs/h_beta_3lvl_relative_difference_responses_intensity.eps}
 \caption{Analytical response function (left) for emergent intensity for H$\beta$ line and the relative difference between the analytical and numerical approach, normalized with respect to the maximum response, for the 4-level hydrogen atom (log scale).}
 \label{example_2}
\end{figure*}

These two tests both show excellent agreement between numerically (i.e. using finite differences) and analytically computed response functions. Note again that we have so far considered pure line transfer and, in that case, we are able to, without much complication, write analytical expressions which are exact. It is then no surprise then our results show such a high level of agreement with numerically computed response functions. 


\subsection{Calcium II 8542 line}

Infrared line of singly ionized Calcium at 8542\,$\AA$ is one of the top contenders for chromospheric diagnostics \citep[see][for an in depth discussion of response function and diagnostic capabilities]{Ca_diag_Jaime}. This is also one of the lines that NLTE inversion code NICOLE can routinely handle. As noted in the introduction, NLTE inversions available at the moment are cumbersome, which led some authors to use approximate approaches \citep{Beck_inversion_2015}. An ideal option would be to have a diagnostics based on response functions which are not computed numerically but in a faster way. We have illustrated accuracy of our approach in previous two subsections and now we employ it to a more realistic case of Ca 8542 line formation. 

We again consider FALC atmospheric model, without magnetic or velocity fields and a 5-level Calcium atom, as considered in \citet{Ca_diag_Jaime}. For the simplicity, we do not consider line asymmetry due to presence of isotopes, but this effect should be straightforward to include in our approach. Contrary to previous subsections, we now also consider other sources of opacity, namely: H- bound-free and free-free opacity, electron scattering, Rayleigh scattering on neutral hydrogen, and bound-free opacity of Ca atom. We simplify problem a bit by considering electron density and its derivative to be equal to their LTE values. 

Obviously, in this case the method presented in \ref{nlte_lvls} has to be modified to account for continuum radiation. By accounting carefully for the sources of opacity and emissivity and their derivatives, both the coupling matrix which describes coupling between all levels and all points in the atmosphere and the right hand side of Eq.\,\ref{final_linear_system} can be properly modified and level responses can be obtained. This process is a bit cumbersome but straightforward. Again, after obtaining level responses, emissivity and opacity perturbations are easily computed and propagated to get responses of the emergent intensity.

\begin{figure*}
 \includegraphics[width = 0.5\textwidth]{figs/ca_5lvl_spectrum.eps}
 \includegraphics[width = 0.5\textwidth]{figs/ca_5lvl_numerical_responses_intensity.eps}
 \includegraphics[width = 0.5\textwidth]{figs/ca_5lvl_analytical_responses_intensity.eps}
 \includegraphics[width = 0.5\textwidth]{figs/ca_5lvl_relative_difference_responses_intensity.eps}
 \caption{Intensity response functions for a 8542 Ca line in FALC model atmosphere. Top left: Emergent line profile; Top right: Numerical (finite difference) computation of intensity responses; Bottom left: Responses computed analytically using the method explained in the paper; Bottom right: Relative differences, normalized with respect to the maximum response, between analytical and numerical computations (in log scale).}
 \label{example_3}
 \end{figure*}

Fig.\,\ref{example_3} shows emergent line profile and response function to temperature, computed with finite difference and in analytical way. Our emergent profile is somewhat different from one shown in \cite{Ca_diag_Jaime}. This difference could be due to the absence of the magnetic field (above authors consider 500\,Gauss magnetic field), or because of different treatment of collisional rates. However, this difference is not crucial for the sake of this paper as we are more interested in the next three panels. We see that intensity response function computed with analytical and finite-difference approaches are practically indistinguishable and that relative differences are really small. Again, our response functions are not identical to ones shown in \cite{Ca_diag_Jaime}. Namely, we lack the negative ``dip'' very high in the atmosphere (see their Fig.\,6). However, this difference is due to difference in models, and not the shortcoming of the method presented here.  

%\section{Response functions to velocity}

\section{Discussion and Conclusions}

In this paper we have outlined an analytical approach for the computation of the intensity response functions for lines formed in non-local thermodynamic equilibrium (NLTE). We take the derivatives of the rate equations analytically and then follow them through to get a large linear system where matrix on the l.h.s. describes the coupling between levels and points in the atmosphere (and hence has the dimension of $NL\times ND$) and the r.h.s. describes ``known'' perturbations (e.g. derivatives of line profile, collisions, etc.). The matrix does not depend on the perturbation itself so it can be decomposed once and then used to obtain responses to level perturbations of different model parameters at different depths. Once responses of level populations are known, we use them to compute appropriate perturbations in opacity and emissivity and then propagate those to get the response of emergent line profile. 

In this paper we have restricted ourselves to the scalar (i.e. unpolarized) case and to the response functions to temperature. We have considered three examples: A pure line transfer for a prototype 2- and 4-level atoms, and a 5 level Ca\,II atom where we focused on the response of Ca\,II\,8542 line. For all the cases considered we obtained excellent agreement between intensity response function computed using numerical (finite-difference) and analytical approach. The advantage is that the analytical approach is much faster. For our specific implementation, it takes less time than \textbf{one} NLTE solution, while finite difference approach needs  ${\mathcal O}(ND)$ NLTE solutions where $ND$ is number of depth points in the atmosphere. 

While this work has its own interesting aspects from the point of theoretical radiative transfer, its main application should be NLTE inversion. Current state-of-the-art codes use numerical response functions, which are time consuming. To compute responses to model parameters in NLTE inversions, one needs at least one (preferably two, for better numerical derivatives) NLTE solution per model parameter. This means at least an order of ten NLTE solutions for one set of derivatives, that is for one iterative step in minimization procedure. Our method would require time similar to one NLTE solution and would thus offer an order of magnitude acceleration. The disadvantage, if one can refer to it that way, is that our formulation is somewhat complicated and that proper implementation requires knowledge of intricacies of NLTE radiative transfer. Also, we can foresee problems where completely consistent treatment will be cumbersome (e.g. overlapping lines of different elements which are all in NLTE). Finally, the inversion of the coupling matrix (Eq.\,\ref{final_linear_system}), must be stable and reliable. The method of choice for that is, without doubt SVD decomposition as it will help us identify the responses which are not well-constrained. All that said, there are cases where our method should be very useful. Namely, unblended, strong NLTE lines formed in relatively simple atomic models (i.e. with small number of levels). We have illustrated this on an example of 8542 line of CaII.

In future years we plan to implement this approach in an NLTE inversion code, but also to perform some purely theoretical investigations. Namely, to investigate responses of level themselves to velocity and magnetic fields and study level and point coupling (coupling matrix itself is an interesting object to study). Finally, this method allows straightforward evaluation of the response of radiation anisotropy to the atmospheric parameters which is a first step toward node-based Hanle inversion, which would be a new step in solar spectropolarimetry. 

\begin{acknowledgements}
 We thank Smitha Narayanamurthy and Rafael Manso Sainz for stimulating and useful discussions. 
\end{acknowledgements}

\appendix
\section{Computing intensity responses to opacity and emissivity perturbations}
\label{Appendix_A}

A necessary ingredient in the computation of the response function for intensity is the computation of the derivative of specific monochromatic intensity in a given point with respect to opacity and emissivity everywhere in the atmosphere, that is:
$$\frac{\partial I_l}{\partial \chi_l'}$$ 
and 
$$\frac{\partial I_l}{\partial \eta_l'}.$$ 
For the clarity, we have omitted dependence on direction and wavelength. Specific intensity depends on emissivity and opacity everywhere in the atmosphere through the process of formal solution, which has to be numerical. This means that the value of the intensity will depend on the employed numerical scheme as well. This is an extremely important point to understand. The fact that different solvers for radiative transfer equation yield different results also means that use of different solvers will result in different inversion results. We can only hope that our solver is accurate and that it's precision with respect to the ``true'' solution is good enough, when compared with observational uncertainties.

After adopting a specific formal solver, the task is to compute the response of intensity in the point $l$ to an infinitesimal perturbation of opacity or emissivity in the point $l'$. \citet{dtibook} proposes an analytical approach, which, in the scalar case, yields:
\begin{equation}
 \frac{d\delta I}{ds} = -\chi \delta I + \eta_{\rm{eff}},
 \label{a_pert}
\end{equation}
where:
\begin{equation}
\eta_{\rm{eff}} = \delta \eta - \delta \chi I
\end{equation}
where prefix $\delta$ refers to perturbed quantities. Eq.\,\ref{a_pert} is then solved by a suitable numerical method, mostly likely by the same one we are using to solve the radiative transfer equation. This approach would be exact if the numerical solution would be infinitely precise. It is, however, not as integrating different functions on a discrete grid yields different inaccuracies. 

To maximize the precision, we have went through the whole process of numerical formal solution and systematically found the derivative of each expression, and then propagated it through. For this work we use second order Bezier solver as in \citet{JaimeBezier}. In this approach, intensity in point $l$ is computed as:
\begin{equation}
 I_l = I_{l-1} e^{-\Delta} + w_{l-1}(\Delta)S_{l-1} + w_{l}(\Delta) S_l + w_C(\Delta) C
 \label{bezier}
\end{equation}
where $\Delta=|\tau_l - \tau_{l-1}|$ and $C$ is so called control point which is computed from derivative of the source function with respect to the optical depth. 

Given arrays containing values of $\chi$ and $\eta$, for a particular direction and wavelength, it is necessary to first compute optical depth scale, source function, and source function derivative with respect to optical depth, and then to employ Bezier solver. The computation of the responses of the intensity with respect to opacity and emissivity has few additional steps and requires derivatives of all quantities in Eq.\,\ref{bezier} with respect of opacity and emissivity. The process consists of the following steps:
\begin{enumerate}
 \item Given opacity, numerically compute its spatial derivative, needed to perform spatial integration of opacity, i.e. to get optical depth.
 \item Compute response of the opacity derivative to unit perturbations in opacity at each point.
 \item From the values of opacity and spatial opacity derivative compute optical depth scale. 
 \item From the perturbations of opacity and perturbations of opacity derivative, compute perturbations of optical depth scale. 
 \item From emissivity and opacity, compute the source function ($S$) and perturbations to the source function.
 \item From the values of the source function, compute derivative of the source function with respect to optical depth ($dS/d\tau$).
 \item From the perturbations of the source function and the perturbations of the optical depth, compute the perturbations of the derivative of the source function.
 \item Using $S$ and $dS/d\tau$, following Bezier scheme, formally solve radiative transfer equation to obtain specific monochromatic intensity at each point in the atmosphere.
 \item Finally, using responses of $S$ and $dS/d\tau$, compute the the response of the intensity at each point in the atmosphere to perturbations of opacity and emissivity in each point of the atmosphere. 
\end{enumerate}

The above procedure is cumbersome, especially once translated in actual equations, but straightforward and it yields responses ($\partial I_{l}/\partial \chi_{l'}$ and $\partial I_{l}/\partial \eta_{l'}$) which agree to ones computed using finite differences down to the level of non-linearities used in the finite differences. After these are computed, response of the intensity at any point to the any combination of perturbations in opacity/emissivity can be found easily:
\begin{equation}
 \delta I_l = \sum_{l'} \left [ \frac{\partial I_l}{\partial \chi_{l'}} \delta \chi_{l'} + \frac{\partial I_l}{\partial \eta_{l'}} \delta \eta_{l'} \right ]
\end{equation}
We use this method both in the construction of the coupling matrix (i.e. when computing level responses) as well as for propagating opacity and emissivity perturbations, once level perturbations are known. This method provides very precise response functions in LTE case as well (in the case of LTE, level responses can be found immediately), and it is quite possible that it will enable better converging inversion schemes for LTE as well.






 



\bibliographystyle{aa}  % A&A bibliography style file (aa.bst)
\bibliography{responses} % your references in file: Yourfile.bib


\end{document}
