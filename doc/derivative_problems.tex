\documentclass[a4paper,10pt]{article}
\usepackage[utf8]{inputenc}
\usepackage{amsmath}

%opening
\title{Total derivative of I with respect to T}
\author{XXX XXX}

\begin{document}

\maketitle

\begin{abstract}
XXX XXX
\end{abstract}

As the issue from Monday kept me thinking two full days I think it deserves to be put on paper, simply to clear up confusion. So, given $h$ and $T$ grids (arrays), which do not depend on each other, and come very complicated mapping $I(T,h)$ (which is actually complete NLTE solution) we are able to compute $dI / dT_i$. However, we have the following issue: for the purpose of inversion, we can to use $\tau$ as our independent grid. If $\tau$ and $T$ grids are given as the input to the problem, we have:
\begin{equation}
 h = h(T,\tau)
\end{equation}
\begin{equation}
 I=I\left (T, h(T,\tau) \right )
\end{equation}
In this case, we have:
\begin{equation}
 \frac{dI}{dT_i} = \frac{\partial I}{\partial T_i} + \sum_j \frac{\partial I}{\partial h_j} \frac{\partial h_j}{\partial T_i}.
\end{equation}
While the derivative of height to the temperature can be found relatively straightforwardly, the derivative of intensity with respect to height grid is pretty non-trivial especially in the NLTE case, as height grid influences both the formal solution (directly) and level populations (indirectly). 

A question I immediately asked myself was: should things not be simpler if I just set $\tau$ (actually, this is $\tau_{500}$), as the independent variable?
\begin{align}
 \frac{dI}{dT_i} = & \frac{\partial I}{\partial T_i} + \sum_j \frac{\partial I}{\partial \tau_j} \frac{\partial \tau_j}{\partial T_i} = \\
 & \frac{\partial I}{\partial T_i} + \sum_j \sum_k \frac{\partial I}{\partial \tau_j} \frac{\partial \tau_j}{\partial \chi_k} \frac{\partial \chi_k}{\partial T_i} 
\end{align}

While derivatives of $\tau$ with respect to opacity are easy, and the derivative of opacity with respect to temperature we already have, the derivative of intensity with respect to $\tau$ is equally non-trivial as before as $\tau$, again, influences both the formal solution and the level populations. 

I come back, again, to the convergence problems in SPINOR, is it possible that this second term here has not been taken into account? I did not see any discussion of this kind in any of the papers, nor in the book of Jose Carlos. I discussed a bit with Anusha and I will discuss with Rafa, but is seems to me that in the completely consistent approach, computing the derivative of $I$ with respect to either $h$ or $tau$ grid is totally unavoidable. Or am I missing something completely? 








\end{document}
